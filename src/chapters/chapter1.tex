\chapter{Introduction}
\label{chap:intro}
% goal
% existing solutions
% outline
    From the very beginning of mankind we assembled, compiled and analyzed the information about our ancestors. The study of genealogy
    is regarded as a noble enterprise, often pursued with extreme care and diligence. Not only it satisfies the desire to carve out a
    place for one's family in the larger historical picture, but also provides the sense of responsibility to preserve the past for
    future generations. However, as a genealogist goes deeper into the family history, its tree grows exponentially, thus making his
    work more and more arduous.

    With the advent of computers, we are able to manage genealogies of incredible size. Now, due to the progress in storage
    engineering, it became possible to maintain and expand existing ancestries of considerable size. But merely keeping the data on
    physical disks is not enough. To sufficiently realize the full potential of computers in genealogy management, one should
    also provide means of inquiry in ancestral data.

    If we want to teach computers understand lineage, we need to construct some type of artificial language that will allow us to
    effectively navigate and query any possible family tree. But observe, that an ancestry already has its own idiosyncratic
    terminology and grammar, which can be successfully used as a natural basis for such a language. Our research is an attempt to do
    exactly that.

    There is another important concept to consider when working with family trees, namely the concept of \textit{time}. A genealogy
    can exist only in specific time framework that is imposed on it by the very nature of history itself. As a result, any computer
    representation of an ancestry that lacks this framework is exorbitantly inadequate. Therefore, its preservation is a crucial
    feature for any software that is aimed for effective genealogy management.

\section{The Goal}
    Our main goal is to design and implement a programming language that will allow us to \textit{efficiently} query temporal genealogical
    trees, while at the same time being as natural as possible. Secondary goal is to develop a software that effectively utilizes this
    language to provide a convenient user interface for working with ancestries.

    There are three requirements that we want our language to satisfy:
    \begin{enumerate}
        \item{\textbf{Expressiveness}. The language should allow for any possible consanguine as well as affinal relations to be
            described.}
        \item{\textbf{Speed}. The response time must not exceed the standard for an interpreted language.}
        \item{\textbf{Simplicity}. Language should be able to express natural kinship and temporal terms as straightforward as possible.}
    \end{enumerate}
    Here the phrase "response time" stands for the time passed between the start and finish of a programs evaluation.
    The last quality is what truly distinguishes our approach from the rest, allowing for the most obvious representation of
    genealogical and temporal information.

\section{Existing Solutions}
    Of course, there are many applications available for working with family trees. According to the web resource\cite{gensoft} there
    are at least $1015$ programs for genealogical management. One may say that given such an abundance of software, there is
    nothing more to add. However, most of it accomplish goals that are drastically different from ours. Moreover, we shall see that
    popular solutions are over-engineered and suffer from what is known as \textit{scope} and \textit{feature creep}, which means
    that the software is overfilled with functionality and it goes far beyond its original scope.

    \begin{enumerate}
        \item{\textbf{Family Tree Builder}. Started as \textit{Software as a Service} in 2003, now became one of the most
            widespread\cite{fambuild} applications in this area. Has a lavish set of functionality, including relative and history
            matching, detailed privacy, family sites, maps and many more. Judging from the image \ref{fig:ftmShot} of its main screen
            we can say that it is a Microsoft Office of genealogy software.}
        \item{\textbf{Neo4j}. A graph database management system\cite{neo4j} written in Java. Initially released in 2007, it is now considered
            to be the most popular graph database according to DB-Engines ranking\cite{rank}. It is feasible, although unacceptable to
            a non-specialist, to create a temporal \textit{ontology} of a family tree using Neo4j and its domain specific language
            Gremlin.  However, any such attempt would fail to take advantage of the structure of kinship language, due to Gremlin's
            general nature.}
        \item{\textbf{RootsMagic}. This application\cite{roots}, similarly to Family Tree Builder, initially was released in 2003, it
            ships with full package of diverse features, such as single-file database, four navigational views, and many more.}
    \end{enumerate}
    During our review of equivalent software, we found out that existing solutions for ancestry management are either inadequate in
    one or more ways, or accomplish quite different goals. There is no application available that would sufficiently achieve our
    main aim with all its stated qualities.

\section{Thesis Outline}
    In the following five chapters we will review the existing literature concerning our topic, discuss the methodology of our work,
    present the project's implementation, demonstrate its correctness and conclude the results.

    In the second chapter "Literature Review" we examine scientific articles, books and journals from three major categories:
    Knowledge Representation, Ontology Building and Natural Language Processing. We focus our attention primarily on the first two,
    since the study of lineage is a special case of knowledge management. Moreover, we inspect various anthropological sources, such
    as works of Claude Levi-Strauss\cite{strauss}, which concerns the nature of kinship structures in different societies.

    In the third chapter "Methodology" we formalize the kinship language of one particular culture, namely traditional American
    culture in the sense of Read\cite{read}. Then we explore its many mathematical properties and show how it can fit into the larger
    framework of a programming language.

    In the forth chapter "Implementation" we present the composition of the complete project, including full documentation of our new
    LISP-like programming language. In particular, three system's components, namely Virtual Assistant, Genealogy Manager and
    Interpreter are discussed.

    In the fifth chapter "Evaluation and Discussion" we show how our application was tested. We use three methods to perform the
    audition: Assertion, Unit and Performance Testing, in which we benchmark the interpreter for our new programming language against
    the interpreter of Racket.

    Finally, in the last sixth chapter "Conclusions" we compile and demonstrate the results of our work. Specifically, we confirm that
    our system indeed achieves the main goal of this thesis and satisfies all proposed qualities.
