\chapter{Conclusion}
\label{chap:conclusion}
% какая была проблема
% значимость
% как мы её решили
% с какими трудностями столкнулись
% что ещё можно сделать
In this work we solved the problem of efficient navigation in temporal genealogies by designing domain-specific programming language
KISP and implementing its interpreter. To further facilitate the use of our program, tree visualiser and virtual assistant have
been developed. By the term "efficient" we mean three things:
\begin{enumerate}
    \item{It should have high expressive power.}
    \item{It must be fast enough.}
    \item{It should be as intuitive as possible.}
\end{enumerate}
We are certain that our project fully covers every one of them.

From the very start of our history, we gathered, analysed and composed information about our ancestors and relatives.
Although, with the ubiquitous use of computers, we can do it more effectively than ever, the present market solutions are
found to be inadequate in one way or the other. Particularly, the vast majority of them are either lacking expressiveness or are
too complicated and, therefore, require a special training just to be used. In contrast, our project manages to be a user-friendly
application, while at the same time having a high level of expressive power.

During the course of our work we faced such challenges as reducing long kinship terms and developing a swift 2D graphical
engine, all of which were successfully solved.

\section{Future Work}
However, there are some topics yet left to tackle in the area of kinship and genealogy management. On the theoretical side, there
is a problem of total term reduction and formal language enrichment. It is also interesting to shift attention to other languages
and cultures with different kinship structures, such as Russian or Hawaiian. The constructed formalism can be considered from the
algebraical side, focusing on its many mathematical properties as a special type of an algebraic system.

On the practical side, one can consider to improve the virtual assistant component. Besides already mentioned Voice Generation \&
Recognition technology, it can be made context-aware, which will increase its intelligence. Additionally, the family ontology can
be enhanced to incorporate information about divorces and deaths.

Further improvements may also include new data types and standard functions for KISP language. Specifically, it is beneficial to
add a \texttt{char} type that represents individual characters in a string. Another useful feature is support for \textit{variadic
lambdas}, which will significantly increase the versatility of KISP.

Moreover, one can also consider including capabilities for a logical reasoning into KISP. They will be applicable for inferring
implicit time constraints for events, whose exact date is unknown. For instance, if we are uninformed about a birthday of a
person, but we do know his parents and his children birthdays, we can justifiably bound this missing date to a specific time
interval.
