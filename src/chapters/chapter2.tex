\chapter{Literature Review}
\label{chap:lr}
\chaptermark{Second Chapter Heading}
The purpose of this chapter is to give a brief survey of academic literature with respect to our thesis.
There are two major areas of Computer Science that are related to out research: Knowledge Representation (KR) and Natural Language
Processing (NLP). We will examine them one by one and highlight some of the most important works.

\section{Knowledge Representation}
The field of Knowledge Representation is concerned with how the knowledge about our physical world can be stored, managed and
utilized by computers. KR owes its existence to the more general field of Artificial Intelligence, which prompted the study of
encoding information about the physical reality into an intelligent system in such a way that it can be used by that system to
solve complex problems.

The main presupposition of the whole field of KR is that, in order for an intelligent agent to resolve a difficult problem, it
needs an access to some form of a knowledge specific to a particular domain and that knowledge should be stored inside the agent.
This presupposition is now widely known as the \textit{Knowledge Representation Hypothesis}:
\begin{quote}
    Any mechanically embodied intelligent process will be comprised of structural ingredients that (a) we as external observers
    naturally take to represent a propositional account of the knowledge that the overall process exhibits (b) independent of such
    external semantical attribution, play a formal but causal and essential role in engendering the behavior that manifests that
    knowledge.
\end{quote}
This original formulation of the hypothesis is due to Smith \cite{smithyp}.

Brachman and Levesque outlined the hypothesis in their article \textit{Expressiveness and tractability in Knowledge Representation
and Reasoning}\cite{krhyp}. The authors argue that the trade-off between \textit{expressiveness} of a knowledge-based system and
its \textit{tractability} (i.e. the ability to reason correctly) is intrinsic to every such system, and can only be partially
solved.  According to them, it is impossible to implement a knowledge-based system that will be both highly expressive and
completely tractable.

Moreover, the whole enterprise of encoding knowledge directly into an agent is proven to be useful only in domain-specific
applications, such as the famous \textit{block world} \cite{shrdlu} domain. A solution with at least partially hard-coded
knowledge is tremendously difficult to scale. Since the dawn of Machine Learning, the plausibility of the hypothesis is
continuously challenged. Indeed, it is questionable whether we can say about a neural network that it stores knowledge in the
weights of its neurons. Thus, today the field of KR does not enjoy that much popularity because the main focus of the AI community
has shifted to other areas, such as Deep Learning.

The most prominent advances in the KR field were the development of Description (Terminological) and Temporal logics and the
formulation of the concept of Ontology. They all are of great importance to our research, so we will survey them one by one.

\subsection{Ontologies}
The word "ontology", derived from the Greek word meaning "the study of being", was unwarrantably borrowed by computer scientists
from the namesake branch of philosophy concerned with the nature of reality. Despite being a useful coinage in
informatics, not all philosophers are content with such state of affairs \cite{confusion}.

The term "ontology" in Computer Science refers to the mechanism by which reality is compartmentalized into a strictly-defined
categories only to be read by machines later. According to Josephson et al. \cite{ontowhy} an ontology comprises a body of
knowledge about a particular domain of interest. However, we must distinguish between an abstract conceptualization of a
particular domain and a concrete instantiation of it. The latter is usually implied when the plural word "ontologies" is used.

The typical ontology consists of:
\begin{enumerate}
    \item{A finite set of \textit{concepts} (a.k.a. nodes, classes). Represents entities of a domain.}
    \item{A finite set of \textit{properties} (a.k.a. attributes, slots, roles). Represents what can be asked of a concept.}
    \item{A finite set of \textit{relationships} between concepts.}
    \item{A finite set of logical \textit{constraints}, which put the boundaries around what can and cannot be stored in an
        ontology.}
\end{enumerate}
At the first sight, the description of an ontology highly resembles that of a database. Indeed, it is true that every database can
be seen as a special case of an ontology, but not vise-versa. As noted in \cite{krhyp}, the power of ontologies lies not in what can
be said in them, but exactly \textit{what can be left unsaid}. For example, suppose we want to store the birth date of our
grandfather, but all we know about him is that he won a medal fighting in WWII.
Then we are forced, using a database, to left the \texttt{birth\_date} field empty, thus loosing the knowledge of his heroic deed.
But, we can eloquently express this knowledge in some ontological lisp-like language as:
\begin{verbatim}
    (set birth_date (father (father me)) (during WWI))
\end{verbatim}
Later we can use this fact to reason about our ancestor more efficiently.

Ontologies find their natural application in the context of our thesis.
Since the original formulation of a concept, a lot of software has been developed to manage ontologies, including such systems as
Protege, In4j and others. These systems have already been heavily used in the variety of different fields.

For instance, Tan Mee Ting \cite{tanfamily} designed and
implemented a genealogical ontology using Protege and evaluated its consistency with Pellet, HermiT and FACT++ reasoners. He
showed that it is possible to construct a family ontology using \textit{Semantic Web} \cite{web} technologies with full capability of
exchanging family history among all interested parties. Hovewer, he did not adress the issue of navigating the family tree using
kinship terms.

An ontology can be used to model any kind of family tree, but the problem arises when a user wants to query his relatives using
kinship terms. No standard out-of-the-box ontology query language is able to articulate statements such as in our example above.
Although an ontology can be tailored to do so, it is not in any way a trivial matter. Maarten Marx \cite{xcpath} addressed
this issue, but in the different area. He designed an extension for XPath, the first order node-selecting language for XML.

Catherine Lai and Steven Bird \cite{ling} described the domain of linguistic trees and discussed the expressive requirements for a
query language. Then they presented a language that can express a wide range of queries over these trees, and showed that the
language is first order complete. This language is also an extension of XPath.

Artale et al. \cite{artale} did a comprehensive survey of various temporal knowledge representation formalisms.
In particular, they analysed ontology and query languages based on the linear
temporal logic LTL, the multi-dimensional Halpern-Shoham interval temporal logic, as well as the metric temporal logic MTL. They
note that the W3C standard ontology languages, OWL 2 QL and OWL 2 EL, are designed to represent knowledge over a static domain,
and are not well-suited for temporal data.

Modelling kinship with mathematics and programming languages, such as LISP, has been an extensive area of research.
Many people committed a lot of work into the field, including Bartlema and Winkelbauer, who investigated\cite{bartlema} the
structure of a traditional family and wrote a simple program that assigns fathers to children. Their main purpose was to
understand how this structure affects fertility, mortality and nuptiality rates. Although promising, small step has been made
towards designing a language to reason about kinship. Also, their program cannot express temporal information.

Another prominent attempt in modelling kinship with LISP was made\cite{findler} by Nicholas Findler, who examined various
kinship structures that exist in literature and combined them together to create a LISP program that can perform arbitrary complex
kinship queries. Although his solution is culture-independent, he did not take the full advantage of LISP as a programming
language, and because of that it is impossible to express queries which are not about family interrelations. In contrast, our system
does not suffer from that restriction.

More abstract, algebraic approach was taken\cite{read} by D. W. Read, who analysed the terminology of American Kinship in terms of
its' mathematical properties. Specifically, he invented an algebraic system, a semigroup, isomorphic to the one particular type of
kinship: American Kinship, which refers to the terminology used by English speakers. His algebra clearly demonstrates that a
system of kin terms obeys strict rules which can be successfully ascertained by formal methods.

Periclev et al. developed\cite{vlad} a LISP program called \texttt{KINSHIP} that produces the guaranteed-simplest analyses,
employing a minimum number of features and components in kin term definitions, as well as two further preference constraints that
they propose in their paper, which reduce the number of multiple componential models arising from alternative simplest kin term
definitions conforming to one feature set. The program is used to study the morphological and phonological properties of kin terms
in English and Bolgarian languages.

According to many authors\cite{read} \cite{vlad} there have been many attempts to create adequate models of kinship based on
mathematics, but these models are either lacking management of temporal information or use only a limited subset of their
programming language. This is not the case with our system.

\subsection{Temporal and Description Logics}
The usage of logic in the field of KR is motivated by its excellence in such areas as mathematics and computer science in general.
The early researches in KR saw the unharvested power of logic -- especially first-order logic -- as a main component in any
intelligent system. Subsequent works showed that FOL can provide semantics for specific kind of KR structures:
\textit{frames} \cite{frames}. Later, Brachman and Levesque proved \cite{krhyp} that we do not need the \textit{whole} FOL for that
purpose, but only certain fragments of it. Moreover, different fragments of FOL have different expressive power and tractability.
Thus, a research began under the label \textit{terminological systems}, only to be later renamed to
\textit{Description Logic} when the main focus was shifted to the properties of underlying logical systems.

Description Logic finds its application in the context of this thesis as a natural formalism for family trees. However, as
expressive as any DL can be, formulating the concept of time requires adding another modal operator. Any logic which handles time
is knows as \textit{temporal logic}. Philosophers have tried to put time into a coherent framework since Aristotle, in
the 20th century mathematicians and computer scientists proposed various formalisms, among which was Allens' \textit{interval
algebra}\cite{allen} and the temporal logic of Shoham\cite{shoh}. Thus, what we need in this thesis is an amalgamation of a
description and temporal logic.

The first successful attempt at integrating two logics is due to Schmiedel \cite{schmiedel}. He combined the DL in the tradition
of KL-ONE \cite{klone}, Shohams' \cite{shoh} temporal logic and Allens' \cite{allen} algebra into one unifying framework. The main
features of his formalism are the complete preservation of original DL and the use of lisp-like syntax for expressing roles,
concepts and time entires.

The application of temporal logic to graphs, relational databases and ontologies is also a heavily-invested subject.
Barcelo and Lubkin examined \cite{barcelo} several temporal logics over unranked trees and characterized commonly used fragments of
first-order (FO) and monadic second-order logic (MSO) for them. They also considered MSO sibling-invariant queries, that can use the
sibling ordering but do not depend on the particular one used, and captured them by a variant of the $\mu$-calculus with modulo
quantifiers.

Alexander Tuzhilin and James Clifford defined \cite{clif} a temporal algebra that is applicable to any temporal relational data
model supporting discrete linear bounded time. This algebra has the five basic relational algebra operators extended to the
temporal domain and an operator of linear recursion. They showed that this algebra has the expressive power of a safe temporal
calculus based on the predicate temporal logic with the "until" and "since" temporal operators.

Perry in his dissertation \cite{perry} highlighted that even in state-of-the-art ontological query languages, such as OWL,
expressing the concept of time is an arduous task. He augmented the \textit{Resource Description Framework} with temporal RDF
graphs and extended the W3C-recommended SPARQL query language to support these new structures.

An adequate representation of time is the holy grail among researchers in the field of ontology development.  Baratis et al.
\cite{toql} designed and implemented \textit{TOQL}: a high-level SQL-like language which is capable of expressing temporal
queries. They motivate the need for such a language by noting that conveying the concept of time using classical languages, such
as OWL, is proven to be difficult, although feasible. They also developed an application that supports translation and execution
of TOQL queries on temporal ontologies combined with a reasoning mechanism based on event calculus.

\section{Natural Language Processing}
The main goal of Natural Language Processing (NLP) field is to invent, study and implement algorithms and techniques that help a
computer understand an ordinary language, such as English, Russian, French or Swahili.

A lot of research in NLP is dedicated to the problem of querying a relational database in some natural language. Since the early
developments, a substantial progress has been achieved. For instance, Jeremy Ferrero et al. proposed and implemented \cite{fr2sql}
a solution to query any database, irrespective of its' schema, in virtually any natural language. They showed that it supports
more operations that most of the other translators. They tested their program on English and French languages.

Another similar attempt was made \cite{not} by Norouzifard et al. They implemented an expert system using Prolog to transform a
sentence in a natural language to SQL. Chaudhari \cite{chaudhari} presented a light weight technique of converting a natural
language statement into equivalent SQL statement.

Nelken et al. took \cite{nelken} a step further and presented a novel controlled NL interface to \textit{temporal databases}, based on
translating NL questions into \textit{SQL/Temporal}, a temporal database query language. They noted that their translation method
is considerably simpler than previous attempts in this direction.

\section{Conclusions}
In this literature review we surveyed several major field in Computer Science. We showed that each of these fields has been advanced
considerably over the last half-century, especially the domain of ontologies. However, we did not find a research that would
satisfy all of the following criteria:
\begin{enumerate}
    \item{Employ either a temporal ontology or a temporal database to store knowledge of temporal family relations.}
    \item{Propose a solution for effective navigation in a genealogical tree via kinship terms.}
    \item{Design and implement a text parser for querying temporal ontologies in natural language.}
\end{enumerate}
Although there were articles that partially fulfill some of these requirements, none of them satisfied all.
Moreover, since our solution is specifically tailored to work only on family trees, we can conclude that its' performance is
better than any other general one. This entails novelty of our work.
