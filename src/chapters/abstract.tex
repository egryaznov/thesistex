\begin{abstract}
<<<<<<< HEAD
% give context (why it's important
% in this thesis we...
% present features
% implementation
% future work

We, as human beings, were always interested in the history of our families. From the medieval times the subject of genealogy
began to gain prominence, and now its considered one of the most important areas of history. Today computers play a crucial role
in the modern ancestry management, they are used to collect, store, analyse, sort and display the genealogical data. However,
current applications do not take advantage of the structure of a kinship system, and therefore they are inaccessible for a typical
user.

In this thesis we propose a new domain-specific language, called KISP, based on a formalisation of the English's kinship
system, for accessing and querying traditional genealogical trees. We implement its interpreter together with an ancestry
visualizer and a virtual assistant that helps the user in genealogy management. KISP is a dynamically typed LISP-like programming
language with such features as kinship term reduction and temporal information expression.

Our solution provides the user with a coherent genealogical framework that allows for a natural navigation over any traditional
family tree.

In the future work we would like add support for other, non-English systems of kinship, extend KISP with new features and
make the virtual assistant more intelligent.
=======
>>>>>>> 879ec8c8d5982b1008d17da8b7104f2aec7196a3

\end{abstract}
